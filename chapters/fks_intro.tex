\begin{introduction}
\newpage

\anonchapter{Введение}

Предлагаемое учебное пособие предназначено для студентов института новых материалов и нанотехнологий (полупроводникового профиля) и преподавателей, проводящих лабораторные работы по курсу Физика конденсированного состояния, ч.1 «Электронная структура твердых тел». В лабораторном практикуме рассматриваются методы измерения удельного электросопротивления, типа, концентрации и подвижности свободных носителей заряда. В теоретическом введении анализируются методы управления этими параметрами на основе зонной теории электронного строения кристаллических твердых тел и квантовой статистики.

\end{introduction}