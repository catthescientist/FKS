\newpage

\anonchapter{Лабораторная работа №1}
\setcounter{chapter}{1}

\begin{center}
Измерение удельного электрического сопротивления полупроводников двухзондовым методом\\
(4 часа)
\end{center}

\section{Цель работы}
Определение распределения удельного сопротивления по длине образца двухзондовым методом. Измерения проводятся при комнатной температуре.

\setcounter{chapter}{1}

\section{Теоретическое введение}
\subsection{}
\paragraph{Характеристика удельного сопротивления полупроводников}

Фундаментальным экспериментальным законом, устанавливающим связь приложенного к проводящему образцу напряжения $U$ и протекающим в образце током $I$ является экспериментальный закон Ома:
\begin{equation}
I = \frac{U}{R}
\end{equation}
где $R$ - электросопротивление образца, является константой.

Электросопротивление $R$ зависит от геометрической формы и размеров образца, а также характеристики материала - удельного электросопротивления $\rho$. Для однородного образца правильной геометрической формы длиной $L$ и площадью поперечного сечения $S$
\begin{equation}
R = \rho \frac{L}{S}
\end{equation}

Если в этом случае выразить интегральные характеристики $U$ и $I$ через дифференциальные $j$ (плотность тока) и $\mathcal{E}$ (напряженность электрического поля), получаем закон Ома в дифференциальной форме
\begin{equation}
j = \frac{I}{S}, \mathcal{E} = \frac{U}{L} \rightarrow j = \frac{\mathcal{E}}{\rho} = \sigma \mathcal{E}
\end{equation}
где $\sigma$ - удельная электропроводность вещества.

В свою очередь, $\sigma$ определяется концентрацией свободных носителей заряда (СНЗ) $n$ и их подвижностью $\mu$:
\begin{equation}
\sigma = e n \mu
\label{UES}
\end{equation}

\begin{equation}
\mu = \frac{V_{\text{др}}}{\mathcal{E}}
\label{mu_from_Vdr}
\end{equation}
где $e = 1.6e^{-19}$ Кл - заряд электрона, $V_{\text{др}}$ - средняя дрейфовая скорость движения электрона под действием электрического поля.

Поведение $\sigma$ в кристаллических полупроводниках и металлах различно. Для металлов $\sigma$ - постоянная величина, в то время как в полупроводниках $\sigma$ зависит от примесного состава, кристаллического совершенства материала и внешних факторов — освещения, радиации и др. Эти материалы имеют также различный характер температурной зависимости удельной электропроводности.

Для полупроводников, у которых есть два типа СНЗ - электроны с концентрацией $n$ и дырки с концентрацией $p$ - формулу (\ref{UES}) необходимо дополнить:

\begin{equation}
\sigma = e n \mu_{n} + e p \mu_{p}
\end{equation}
где $\mu_{n}$ и $\mu_{p}$ - подвижость электронов и дырок соответственно.

\paragraph{Концентрация свободных носителей заряда}

Основой для понимания физических процессов и электрических явлений в твердом теле является зонная теория электронных спектров, базирующаяся на квантовомеханичеких представлениях. Концентрация свободных электронов — это концентрация занятых квантовомеханических состояний в зоне проводимости, а концентрация дырок — концентрация незаполненных состояний в валентной зоне. При температуре 0\textdegree К в полупроводнике свободных носителей заряда нет, в то время как концентрация электронов в металле практически не зависит от температуры и составляет величину порядка концентрации атомов металла в единице объема ($\approx 10^{22} \text{см}^{-3}$). СНЗ в полупроводниковых материалах появляются за счет термической генерации свободных носителей заряда (за счет энергии кристаллической решетки). Возможны несколько случаев:
\begin{enumerate}
\item переход электронов из валентной зоны в зону проводимости (в этом случае создаются одинаковые концентрации электронов и дырок $n = p = n_{i}$);
\item переход электронов из валентной зоны на уровень акцепторной примеси $E_{A}$, при этом создаются свободные дырки;
\item переход электронов с уровня донорной примеси $E_{\text{Д}}$ в зону проводимости (создаются свободные электроны).
\end{enumerate}

Верхний предел концентрации СНЗ при комнатной температуре в полупроводниках определяется пределом растворимости легирующих примесей ($\approx 10^{19} \text{см}^{-3}$), нижний предел определяется собственной концентрацией СНЗ - $n_{i}$.

Важнейшим параметром проводящего материала, однозначно связанным с концентрацией ННЗ, является уровень Ферми $F$. Для невырожденного материала
\begin{equation}
n = N_{c} \exp \left( -\frac{E_{c}-F}{k T} \right),
p = N_{v} \exp \left( -\frac{F - E_{v}}{k T} \right)
\end{equation}
где: $k$ – константа Больцмана, $N_{c}$ ($N_{v}$) — плотность состояний на дне зоны проводимости (потолке валентной зоны), зависящая от температуры и эффективной массы соответствующих СНЗ:
\begin{equation}
\begin{split}
%\begin{array}
N_{c} &= 2 \left( \frac{2 \pi m^{*}_{d} k T}{h^2} \right) = 4.82 \cdot 10^{15} \left( \frac{m^{*}_{d}}{m} \right)^{\frac{3}{2}} T^{\frac{3}{2}} = 2.5 \cdot 10^{19} \left( \frac{m^{*}_{d}}{m} \right)^{\frac{3}{2}} \left( \frac{T}{300} \right)^{\frac{3}{2}} \\
N_{v} &= 2.5 \cdot 10^{19} \left( \frac{m^{*}_{pd}}{m} \right)^{\frac{3}{2}/2} \left( \frac{T}{300} \right)^{\frac{3}{2}}
%\end{array}
\end{split}
\end{equation}

Термодинамически равновесные концентрации $n$ и $p$ в невырожденном материале связаны соотношением:
\begin{equation}
n \cdot p = n_{i}^2 = N_{c} N_{v} \exp \left( -\frac{E_{g}}{2 k T} \right)
\end{equation}

При низких температурах доминирует процесс ионизации примеси, в области средних температур, к которой для большинства практически значимых полупроводниковых материалов относится комнатная температура, концентрация СНЗ равна концентрации легирующей примеси, и полупроводник является либо электронным либо дырочным. При дальнейшем повышении температуры примесный полупроводник становится собственным. Температура перехода к собственной проводимости тем выше, чем больше ширина запрещенной зоны и чем выше концентрация легирующих примесей в полупроводнике.

Легирующими или мелкими примесями в полупроводнике, являются примеси, энергия ионизации которых ($E_{c}$ — $E_{\text{Д}}$ для донорной примеси и $E_{\text{А}}$ — $E_{\text{v}}$ для акцепторной) сравнима со средней энергией кристаллической решетки в расчете на один атом — $kT$ ($0.025$ эВ). Для широкого круга алмазоподобных полупроводников такими примесями являются элементы, валентность которых отличается от валентности атомов полупроводникового материала на единицу. Так, для кремния и германия легирующими будут элементы III (акцепторы) и V (доноры) групп периодической системы Менделеева. Для соединений $A_{III}B_{V}$ – элементы II и VI групп. Элементы IV группы могут быть как донорами, так и акцепторами, в зависимости от того, элемент какой из подрешеток они замещают.

\paragraph{Подвижность свободных носителей заряда}

Для объяснения того факта, что электрическое поле вызывает в проводящей среде движение с постоянной скоростью (\ref{mu_from_Vdr}), а не ускорением, как это должно быть при действии силы величиной $e \overrightarrow{\mathcal{E}}$ на частицу с массой $m$, в классической теории электропроводности было введено понятие среднего времени свободного пробега $\tau$ как величины, обратной вероятности столкновения электрона с решеткой. При таком столкновении энергия, полученная от электрического поля, отдается решетки и восстанавливается первоначальный импульс. В этом случае подвижность определяется выражением:
\begin{equation}
\mu = \frac{\mathcal{E} \tau}{m}
\label{mu_from_E}
\end{equation}

В рамках классической физики было непонятно, почему длина свободного пробега $L$ (произведение времени свободного пробега на тепловую скорость носителя заряда) составляет сотни параметров кристаллической решетки, т. е. почему при движении по кристаллической решетке электрон сталкивается только с одним из сотен атомов на его пути. Объяснить этот экспериментальный факт (как и ряд других, связанных в частности, с различием поведения электропроводности металлов и полупроводников) удалось в рамках квантовомеханической теории, точнее, зонной теории кристаллических твердых тел. Электрон как квантовомеханическая ферми-частица обладает не только энергий и импульсом, но и волновыми свойствами. Волновая функция электрона в периодическом поле идеальной кристаллической решетки периодически модулирована с периодом, равным постоянной решётки, что позволяет электрону двигаться без рассеяния. В идеальном кристалле время свободного пробега и подвижность были бы бесконечными. Однако идеальных кристаллических решеток не существует в принципе. При взаимодействии с локальными нарушениями периодического Кулоновского поля, создаваемого ядрами атомов, т. е. с дефектами кристаллической решетки, импульс электрона изменяется. К основным дефектам кристаллической решетки можно отнести тепловые колебания атомов, примесные атомы (нейтральные и ионизированные), дислокации, двойники и т.д.

При приложении электрического поля изменяется заполнение разрешенных квантовомеханических состояний. При выключении за счет взаимодействия с дефектами решетки система релаксирует, переходя в термодинамически равновесное состояние. Поэтому вводится понятие времени релаксации, которое в слабых электрических полях совпадает с введенным в классической теории электропроводности понятием времени свободного пробега и связано с подвижностью СНЗ выражением (\ref{mu_from_E}).

Подвижность СНЗ зависит от среднего времени релаксации возбужденного состояния, которое, в свою очередь, определяется механизмом рассеяния СНЗ. Термин «механизм рассеяния» отражает тот факт, что в результате столкновения с дефектами кристаллической решетки поток электронов (дырок) вдоль направления вектора напряженности электрического поля постепенно уменьшается (рассеивается), что приводит к исчезновению электрического тока после выключения электрического поля. Но процессы рассеяния идут и при протекании электрического тока. Чем больше концентрация дефектов кристаллической решетки, тем быстрее происходит процесс выбывания отдельных электронов из потока, определяющего электрический ток в материале, тем меньше среднее время релаксации, подвижность СНЗ и величина электропроводности. Каждому типу дефекта кристаллической решетки соответствует свой механизм рассеяния.

