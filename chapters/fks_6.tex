\newpage

\anonchapter{Лабораторная работа №6}
\setcounter{chapter}{6}

\begin{center}
Изучение поглощения света в полупроводниках\\
(4 часа)
\end{center}

\section{Цель работы}
Целью работы является изучения основных механизмов поглощения света в полупроводниках и освоение методики исследований спектров поглощения света в полупроводниках в видимой и ближней инфракрасной области спектра.

\section{Теоретическая часть}
Если поток фотонов с длиной волны $\lambda$ и интенсивностью $I_{0}(\lambda)$ падает на плоскопарралельный образец толщиной $d$, то можно исследовать интенсивность отражённого от поверхности образца света $I_{R}(\lambda)$ и определить безразмерный коэффициент отражения

\begin{equation}
R(\lambda) = \frac{I_{R}(\lambda)}{I_{0}(\lambda)}
\end{equation}

а также измерить интенсивность прошедшего через образец света $I_{T}(\lambda)$ и определить безразмерный коэффициент пропускания

\begin{equation}
T(\lambda) = \frac{I_{T}(\lambda)}{I_{0}(\lambda)}
\end{equation}

Интенсивность прошедшего через образец света однозначно определяется коэффициентом отражения, поглощением внутри образца и отражением от внутренних граней. Закон Бугера-Ламберта внутри образца, на поверхность которого падает свет выглядит следующим образом:

\begin{equation}
I(x, \lambda) = I_{0}(\lambda) \left[ 1 - R(\lambda) \right] \exp{(- \alpha (\lambda) x )}
\end{equation}
где $\alpha(\lambda)$ - коэффициент поглощения, имеет размерность обратной длины.

Величина $\alpha^{-1}$ определяет толщину кристалла, проходя через которую свет ослабляется в $e$ раз. Это значит, что можно считать величину $\alpha$ вероятностью поглощения фотона на единице длины. Если имеется несколько незвисимых механизмов поглощения света в кристалле, то полный коэффициент поглощения будет суммой коэффициентов, определяемых каждым механизмом независимо.

\begin{equation}
\alpha(\lambda) = \sum\limits_{i}{\alpha_{i}(\lambda)}
\end{equation}

Зависимость $\alpha(\lambda)$ называется спектром поглощения для данного вещества.